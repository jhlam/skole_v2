\chapter{Future work} \label{futureWork}
Information diffusion and seed selection, in general, computes large graphs, and thus is very time-consuming. There is, therefore, several improvements that have yet to be explored.

\begin{enumerate}
\item \textbf{Different architecture} for this implementation, we used a core including the LFSR; it would be interesting to explore different design. One solution that we did not have the chance to explore is implemented a large buffer connected to a single LFSR that continuously generates a random number. The implemented cores would be then each pop one random number for each coin toss. This solution requires a large buffer and would potentially generate large overhead with reading from the buffer. A large enough buffer would also be required since there are in worst case scenario for a single SpMV run, we would need $n^2$ coin toss. The potential benefits of such a design would be better space utilization. A smaller core would use less resource of the Zedboard. This can result in more parallelization.

\item \textbf{Use larger graph} In graph theory, graph used is often at scale(26-42). The smallest mentioned graph from Graph500, is $2^{26}$, a toy graph. Our graph is not even close to such a large graph. Testing this architecture up against a larger graph would be beneficial. 

\item \textbf{Customize algorithm} As mentioned in Chapter \ref{relatedWork}, HLS can generate a close to hand written design if the algorithm is customized for HLS. An interesting potential improvement would be to analyse algorithm and explore different solutions and implementation.

\item \textbf{Compare different solutions} In this report, we have only shown the result from one architecture, it would be interesting to compare different schemes and other solutions. 

\item \textbf{Memory Optimization}. For this algorithm, we store the entire adjacency matrix. This is inefficient for a sparse matrix. A potential improvement would be to explore a different storage format for the adjacency matrix.

\item \textbf{Try different seed selection algorithm} In this report, we implemented the greedy seed selection. It would have been interesting to compare different seed selection algorithm and compare the results. There are papers \citep{Chen:2009:EIM:1557019.1557047} that proposes an alternative greedy solution.  
\item \textbf{Better optimization with HLS} would result in potentially better utilization of the resources on the Zedboard. Our design was not the most optimal and could only compute small graphs. One future aspect would be to optimize for much larger networks.

\item \textbf{A general IP core} was implemented, but due to time constraints, were not used for this project. The general IP core had a fixed buffer size, but could accept matrix and vector of a larger size. The IP-core would store nodes in the buffer, compute the SPMV, and then send the result back. This solution resulted in massive overhead with transporting data back and forth between PL and PS. This core was not used due to time constraints.
\end{enumerate}


